\item Una caldera de vapor tiene un volumen de agua de $V_{cal}$ a $T_{1}$ y presión atmosférica, cuando la misma está apagada. Se enciende el quemador  y se calienta hasta que se logra en su interior una presión manométrica de vapor $P_2$. A esa presión de saturación, la temperatura de $T_{2}$. Determine qué cantidad de masa de líquido que se evaporo para lograr esa presión. Considerar al vapor de agua como un gas ideal.

\begin{center}
$V_{cal} = 50\,\text{m}^3 \qquad T_1 = 20 \,^\circ C  \qquad P_2 = 10\,\text{bar} \qquad T_2 = 180\,^\circ C  $
\end{center}

\textit{Nota}: En una caldera real, no puede medirse sólo la presión de vapor, sino que se mide la suma de presiones parciales, de acuerdo a la ley de Dalton. Corrija el cálculo anterior considerando que un 10\% de la masa de gas es aire en vez de agua.
