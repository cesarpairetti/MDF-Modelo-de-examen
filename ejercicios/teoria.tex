\begin{itemize}
 \item {\bf Mecánica de los Fluidos:} Estática, Cinemática y Dinámica
    \begin{enumerate}
    
  \item {\bf Hidrostática}
	\begin{itemize} 
	\item Describir el principio general de la hidrostática.
	\item ¿Qué diferencias existen entre las fuerzas de superficie y de cuerpo? Dar ejemplos de cada una.
	\item Explicar mediante un ejemplo el equilibrio relativo en los casos de aceleración lineal constante y
	de rotación a velocidad constante.
	\end{itemize}


      \item La {\bf Ecuación de Bernoulli} se encuentra transcripta en \ref{eq:Bernoulli}.
	\begin{equation}\label{eq:Bernoulli}
	\frac{p}{\gamma}+	%%Energía de presión del fluido
	\frac{v^2}{2g}+	%%Energía cinética
	z=\mathit{cte} 		%%Energía potencial gravitatoria
	\end{equation}

	\begin{itemize}
	\item ¿Bajo qué hipótesis puede utilizarse este modelo matemático?
	\item ¿Qué representa cada término de la ecuación?
	\item ¿Qué representa la ecuación?¿Para qué se utiliza? Dar un ejemplo.
	\item ¿De qué forma se emplea esta ecuación en el cálculo de cañerías?
	\end{itemize}

      \item Dada la ecuación de {\bf Euler} para el momento lineal (transcripta en \ref{eq:Euler}):
      \begin{equation}\label{eq:Euler}
      \rho \frac{D\mathbf{U}}{Dt}=	%%Variación puntual de momento lineal
      \rho \mathbf{g}		%%Efecto de las fuerzas de cuerpo.
      - \nabla p 			%%Efecto de las fuerzas de presiones, fuerzas de superficie.
      \end{equation}
      \begin{itemize}
      \item ¿Bajo qué hipótesis puede utilizarse este modelo matemático?
      \item ¿Qué representa la ecuación?¿Para qué se utiliza?
      \item ¿Qué representa cada término de la ecuación?
      \item Nombrar algún ejemplo de resolución simplificada de esta ecuación.
      \end{itemize}

      Conociendo a su vez la definición de derivada material (\ref{eq:DM}),
      explicar qué tipos de aceleraciones puede experimentar una partícula
      que se desplaza en el campo de velocidad $\mathbf{U}$.

      \begin{equation}\label{eq:DM}
	\frac{Du}{DT}=\frac{\partial u}{\partial t}+u\frac{\partial u}{\partial x}+v\frac{\partial u}{\partial y}+w\frac{\partial u}{\partial z}=
	\frac{\partial u}{\partial t}+\mathbf{U}_\bullet \left(\nabla u \right)
      \end{equation}

      \item Dada la ecuación de {\bf Navier-Stokes} para el momento lineal (transcripta en \ref{eq:NS}):
      \begin{equation}\label{eq:NS}
      \rho \frac{D\mathbf{U}}{Dt}=	%%Variación puntual de momento lineal
      \rho \mathbf{g}		%%Efecto de las fuerzas de cuerpo.
      - \nabla p 			%%Efecto de las fuerzas de presiones, fuerzas de superficie.
      + \mu \nabla^2 \mathbf{U}	%%Efecto de las fuerzas viscosas, fuerzas de superficie. (de Navier-Stokes)
      \end{equation}
      \begin{itemize}
      \item ¿Bajo qué hipótesis puede utilizarse este modelo matemático?
      \item ¿Qué representa la ecuación?¿Para qué se utiliza?
      \item ¿Qué representa cada término de la ecuación?
      \item Nombrar algún ejemplo de resolución simplificada de esta ecuación.
      \end{itemize}

      Conociendo a su vez la definición de derivada material (\ref{eq:DM}),
      explicar qué tipos de aceleraciones puede experimentar una partícula
      que se desplaza en el campo de velocidad $\mathbf{U}$.

      \begin{equation}\label{eq:DM}
	\frac{Du}{DT}=\frac{\partial u}{\partial t}+u\frac{\partial u}{\partial x}+v\frac{\partial u}{\partial y}+w\frac{\partial u}{\partial z}=
	\frac{\partial u}{\partial t}+\mathbf{U}_\bullet \left(\nabla u \right)
      \end{equation}

      
      \item {\bf Teoría de capa límite}
      \begin{itemize}
      \item Explicar a qué se llama capa límite. Definir la condición de no deslizamiento.
      \item Describir los distintos tipos de capa límite posibles.
      \item ¿A qué se llama desprendimiento de capa límite? ¿En qué situación ocurre?
      \end{itemize}

      
      
      \item Las {\bf toberas convergentes-divergentes} son elementos de amplio uso en muchas aplicaciones de Ingeniería. El modelo matemático
      empleado en su cálculo se desarrolla a partir de la teoría de flujo compresible.

	\begin{itemize}
	\item Definir el número de Mach y la clasificación de flujos relacionada a este coeficiente adimensional.
	\item Describir la relación entre las variaciones de presión y velocidad en función del cambio de sección.
	Analizar cómo depende del número de Mach.
	\item Describir las posibles evoluciones del flujo en la tobera convergente divergente.
	¿Qué variable determina el flujo que se desarrolla en la tobera?
	\end{itemize}
    \end{enumerate}
	
\item {\bf Instalaciones y máquinas hidráulicas:}
    \begin{enumerate}
      \item {\bf Flujo en conductos}. Al considerar el flujo de un fluido Newtoniano
      en conductos cerrados.
      \begin{itemize}
      \item ¿Cuáles son las variables de interés en este problema?
      \item ¿Cuáles son los posibles regímenes de flujo?
      \item Estimativamente ¿Cuáles son los perfiles de velocidad y corte en cada caso?
      \item ¿Qué es y de qué manera pueden calcularse la pérdida de carga?
      \end{itemize}

      \item {\bf Turbinas}
      \begin{itemize}
      \item ¿Cuál es la principal función de dichas máquinas hidráulicas?
      \item Mencione algunos parámetros de diseño.
      \item ¿Cuáles tipos de turbinas conoce? ¿De qué depende la elección de una u otra?
      \item ¿Cuáles son las variables de contorno a las que debe ajustarse una turbina?
      \item ¿Cómo y para qué se regula el caudal que ingresa al rotor?
      \end{itemize}

      \item {\bf Bombas centrífugas}

      \begin{itemize}
      \item ¿Cuál es la principal función de dichas máquinas hidráulicas? 
	\item Explicar qué es la curva característica de la bomba y qué parámetros la afectan.
	\item ¿Cómo se interpreta el punto de trabajo?
	\item ¿Cómo se comportan dos bombas en paralelo?¿Y en serie?
      \end{itemize}

      \end{enumerate}
  \end{itemize}
